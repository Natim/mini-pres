%\documentclass[trans]{beamer}
\documentclass{beamer}

\usepackage{natim-beamer}
\usepackage{listings}

\usetheme[shownavigation={false},  % true | false
          logo={images/classemini.png},
          titlepageimage={images/titleimage.png},
          header=utbm,          % fullnav | shortnav | utbm
          dept={Classe Mini 650}
        ]{UTBM}
\usecolortheme{UTBMOfficial}
\usepackage{upgreek}

\title[Course au large]{Aventure et embruns salés}
\subtitle{REST@SEA - Mini 433 - Automne \the\year}
\author{À vos marques, prêts, ramez !}
\institute{Rémy Hubscher}
\date{\today}


\hypersetup{
      pdfpagemode = FullScreen,% afficher le pdf en plein écran
      pdfauthor   = {Rémy HUBSCHER},%
      pdftitle    = {Course au large : Embruns salés, aventures l'horizon !},%
      pdfsubject  = {Alma - Course au large},%
      pdfkeywords = {voile, course, mini, 650},%
      pdfcreator  = {PDFLaTeX},%
      pdfproducer = {PDFLaTeX}%
}

\begin{document}

\selectlanguage{french}

\begin{frame}[plain]
  \titlepage
\end{frame}

\begin{frame}
  \frametitle{Voile légère - Dériveur 420}

  \image{420}{Dériveur 420}{0.3}
  
\end{frame}

\begin{frame}
  \frametitle{Voile légère - Catamaran Hobie Cat 16}

  \image{hobie-cat-16}{Hobie Cat 16}{0.7}
  
\end{frame}

\begin{frame}
  \frametitle{Croisière}

  \image{rm-1070}{Dériveur 420}{0.3}
  
\end{frame}

\begin{frame}
  \frametitle{Course au large}

  L'objectif c'est d'aller vite comme en voile légère ... \pause

  \vspace{6em}

  ... mais d'éviter de se retourner comme en voile légère.
  
\end{frame}

\begin{frame}
  \frametitle{Course au large - Mini 650}

  \begin{columns}
    \begin{column}{0.6\textwidth}
      \begin{center}
        \image{rest-at-sea}{Mini 433}{0.17}
      \end{center}
    \end{column}
    \begin{column}{0.4\textwidth}
      \begin{center}
        \textbf{Mini 433} \\
        \vspace{1em}
        \pause \textbf{REST @ SEA} \\
        \vspace{1em}
        \pause Prototype \\ construit en 2003 \\
        \vspace{1em}
        \pause Vainqueur de la \\ Mini Transat 2005
        \vspace{6em}
      \end{center}
    \end{column}
  \end{columns}
\end{frame}

\begin{frame}
  \frametitle{Course au large - Mini 650}

  \begin{columns}
    \begin{column}{0.4\textwidth}
      \begin{center}
        \image{rest-at-sea-hors-eau}{Mini 433}{1}
      \end{center}
    \end{column}
    \begin{column}{0.6\textwidth}
      \textbf{Mini 650 - Prototype}
      \vspace{1em}
      
      Mât en carbone : 12m\\

      \vspace{2em}

      \pause Coque : 6.50m de long, 3m de large max\\

      \vspace{2em}

      \pause Bout dehors : 3m\\

      \vspace{2em}

      \pause Quille basculante : 2m de tirant d'eau\\
    \end{column}
  \end{columns}
\end{frame}

\begin{frame}
  \frametitle{Course au large - Mini 650}

  Aussi stable qu'un voilier monocoque de croisière ... \\

  \vspace{6em}

  \pause ... quasiment aussi humide qu'un bateau de voile légère.
\end{frame}

\begin{frame}
  \frametitle{Le parcours mini 650}

  \begin{center}
    \image{duo2020}{Duo Concarneau}{0.2}
  \end{center}
  \begin{itemize}
    \item Un programme de courses plus ou moins longues.
    \pause \item Les plus courtes de un ou deux jours
    \pause \item Les plus longues de plusieurs semaines
    \pause \item Certaines en double, la majorité en solitaire
  \end{itemize}
\end{frame}


\begin{frame}
  \frametitle{Le projet}

  \begin{itemize}
    \item En 2019/2020 plus de bricolage que de navigation
    \pause \item Un sport mécanique engagé physiquement
    \pause \item Énormément de chose à apprendre\\ (Météo, Matelotage, Composites, Peinture, Sponsoring)
    \pause \item Un environnement hostile\\ (Humidité, Éléments, Météo, OFNI)
    \pause \item Un projet coûteux et ambitieux
  \end{itemize}

\end{frame}

\begin{frame}
  \frametitle{Les objectifs pour 2021}

  \begin{itemize}
    \item Démonter, poncer et repeindre le bateau cet hiver
    \pause \item Faire un maximum de course en double en 2021
    \pause \item Gagner en fluidité dans les manœuvres et faire baisser le stress
    \pause \item Prendre du plaisir et fiabiliser le bateau
  \end{itemize}
\end{frame}

\begin{frame}
  \frametitle{Ce que m'apprends ce projet}

  \begin{itemize}
    \item Certaines choses sont beaucoup plus dure qu'elle ne paraisse
    \pause \item La performance demande du temps
    \pause \item Ne laisser pas l'argent vous empêcher de réaliser vos rêves
    \pause \item Réussir des choses qui nous paraisse impossible est un sentiment incroyable
    \pause \item Restez toujours dans une dynamique d'apprentissage
  \end{itemize}
\end{frame}

\begin{frame}
  \frametitle{Aventure incroyabe}
  \image{map2020}{Trophée MAP}{0.3}

\end{frame}

\end{document}
